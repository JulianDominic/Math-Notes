\documentclass[../Latex-Setup/setup.tex]{subfiles}
\graphicspath{{./images}}

\begin{document}
\title{Sets}
\author{Julian Dominic}
\date{\today}
\maketitle

\section{Introduction to Sets}

\indent A \textbf{set} is a collection of things. We call these things \textbf{elements}.
In Mathematics, we denote a set by writing elements inside curly brackets $\{a,b,c,d,\dots\}$.
For example, if we want a set to contain $2,4,6,8$ exactly, we can write it as $\{2,4,6,8\}$.
A set has either a \textbf{finite} or \textbf{infinite} amount of elements.
For example, we can have the set of all even integers:

\[\{\dots,-4,-2,0,2,4,\dots\}\]
the ellipsis (dots) here indicates that there are more elements, and since they are indicated at the 'ends' of the sets,
we can say that numbers continues forever in both the negative and positive directions.\par

\indent Two sets are equal if they contain the \textbf{exact same elements regardless of their order}.

\[ \{1,2,3\} = \{3,1,2\} \]
\par

\indent We often let uppercase letters stand for sets. Lets say we have a set $A = \{2,4,6,8\}$, we can have some statements:

\begin{itemize}
    \item $2 \in A$: $2$ is an element of $A$; $2$ is in $A$; $2$ in $A$.
    \item $5 \notin A$: $5$ is not an element of $A$.
    \item $2,4,6,8 \in A$: $2,4,6,8$ are in $A$ -- this is how we can indicate that there are several things in a set.
\end{itemize}
\par

\indent \textbf{Cardinality} is how we can measure the size of the sets which is the number of elements in a set. 
For now, we will only concern ourselves where the sets are finite. If $X$ is a finite set, the cardinality of $X$ is $|X|$.\par

\indent There is a special set called the \textbf{empty set}, represented as $\emptyset$ or $\{\}$.
By its name, an empty set has no elements, thus, the cardinality $|\emptyset| = 0$. The empty set is the only set whose cardinality is zero.
We can think of a set as a ``box''. By this definition, the empty set is just an empty box. But what about $\{\emptyset\}$?
It is a box that contains an empty box, so we can say its cardinality $|\{\emptyset\}| = 1$.\par

\indent In Mathematics, we will often the \textbf{set-builder notation} or some variant of it.
It is used to describe sets that are either too big, too complex or hard to list in standard $\{\}$ notation.

\[X = \{\text{expression}:\text{rule}\} \text{ or } \{\text{expression}|\text{rule}\}\]
this can be read as $X$ is the set of all things (elements) of the form ``expression'' such that ``rule''.
There can be many ways to express the same set. If we want to express the set of all even integers:
\begin{align*}
    E &= \{2n: n \in \mathbb{Z}\} \\
    &= \{n: n \text{ is an even integer}\} \\
    &= \{n: n=2k,k \in \mathbb{Z}\} \\
    &= \{n \in \mathbb{Z} : n \text{ is even}\}
\end{align*}
\par


\section{The Cartesian Product}
\section{Subsets}
\section{Power Sets}
\section{Union, Intersection, Difference}
\section{Complement}
\section{Venn Diagrams}
\section{Indexed Sets}
\section{Sets That Are Number Systems}
\section{Russell's Paradox}
\end{document}