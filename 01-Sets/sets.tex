\documentclass[../Latex-Setup/setup.tex]{subfiles}
\graphicspath{{./images}}
\usepackage{listings}
\usepackage{xcolor}

\definecolor{codegreen}{rgb}{0,0.6,0}
\definecolor{codegray}{rgb}{0.5,0.5,0.5}
\definecolor{bgcolour}{rgb}{0.95,0.95,0.92}
\lstdefinestyle{code}{
    backgroundcolor=\color{bgcolour},   
    commentstyle=\color{codegreen},
    keywordstyle=\color{blue},
    numbers=none,
    %numberstyle=\tiny\color{codegray},
    basicstyle=\ttfamily\footnotesize,
    identifierstyle=\color{black},
    frame=L
}
\lstset{style=code}

\begin{document}
\title{Sets}
\author{Julian Dominic}
\date{\today}
\maketitle

\section{Introduction to Sets}

\indent A \textbf{set} is a collection of things. We call these things \textbf{elements}.
In Mathematics, we denote a set by writing elements inside curly brackets $\{a,b,c,d,\dots\}$.
For example, if we want a set to contain $2,4,6,8$ exactly, we can write it as $\{2,4,6,8\}$.
A set has either a \textbf{finite} or \textbf{infinite} amount of elements.
For example, we can have the set of all even integers:

\[\{\dots,-4,-2,0,2,4,\dots\}\]
the ellipsis (dots) here indicates that there are more elements, and since they are indicated at the `ends' of the sets,
we can say that numbers continues forever in both the negative and positive directions.\par

\indent Two sets are equal if they contain the \textbf{exact same elements regardless of their order}.

\[ \{1,2,3\} = \{3,1,2\} \]
\par

\indent We often let uppercase letters stand for sets. Lets say we have a set $A = \{2,4,6,8\}$, we can have some statements:

\begin{itemize}
    \item $2 \in A$: $2$ is an element of $A$; $2$ is in $A$; $2$ in $A$.
    \item $5 \notin A$: $5$ is not an element of $A$.
    \item $2,4,6,8 \in A$: $2,4,6,8$ are in $A$ -- this is how we can indicate that there are several things in a set.
\end{itemize}
\par

\indent \textbf{Cardinality} is how we can measure the size of the sets which is the number of elements in a set. 
For now, we will only concern ourselves where the sets are finite. If $X$ is a finite set, the cardinality of $X$ is $|X|$.\par

\indent There is a special set called the \textbf{empty set}, represented as $\emptyset$ or $\{\}$.
By its name, an empty set has no elements, thus, the cardinality $|\emptyset| = 0$. The empty set is the only set whose cardinality is zero.
We can think of a set as a ``box''. By this definition, the empty set is just an empty box. But what about $\{\emptyset\}$?
It is a box that contains an empty box, so we can say its cardinality $|\{\emptyset\}| = 1$.\par

\indent In Mathematics, we will often see/use the \textbf{set-builder notation} or some variant of it.
It is used to describe sets that are either too big, too complex or hard to list in standard $\{\}$ notation.

\[X = \{\text{expression}:\text{rule}\} \text{ or } \{\text{expression}|\text{rule}\}\]
this can be read as $X$ is the set of all things (elements) of the form ``expression'' such that ``rule''.
There can be many ways to express the same set. If we want to express the set of all even integers:
\begin{align*}
    E &= \{2n: n \in \ZZ\} \\
    &= \{n: n \text{ is an even integer}\} \\
    &= \{n: n=2k,k \in \ZZ\} \\
    &= \{n \in \mathbb{Z} : n \text{ is even}\}
\end{align*}
\par


\section{The Cartesian Product}

The \textbf{Cartesian product} is an operation where we multiply two or more sets together to produce a new set.
To understand what the elements of this Cartesian product are, we need to know what is an ordered pair.
An \textbf{ordered pair} is a list $(x,y)$ of two things $x$ and $y$, enclosed in parentheses and separated by a comma.
As the name implies, while it may look similar to sets, an ordered pair cares about the order of its elements.
This is to show that we can have two ordered pairs -- $(2,4)$ and $(4,2)$ -- while they have the same elements,
they are not equal because the order of each element is different.\par

The Cartesian product of two sets $A$ and $B$ can be written as $A \times B$. Note, the new set produced is called $A \times B$.
When there is a set, there will be its elements (unless it's the empty set):

\[A \times B =\{(a, b) : a \in A, b \in B\}\]
the elements of the Cartesian product of $A$ and $B$ are ordered pairs of the elements found in $A$ and $B$. For example,
$A = \{k, l, m\}$ and $B = \{q, r\}$, their Cartesian product $A \times B = \{(k,q), (k,r), (l,q), (l,r), (m,q), (m,r)\}$.
Do take note of the order of the elements within the ordered pairs, the elements of $A$ come first, followed by the elements of $B$.\par

If $A$ and $B$ are finite sets, $|A \times B| = |A| \cdot |B|$.\par

A quick note for programmers, if you haven't noticed it already, the Cartesian product is very similar to nested ``for'' loops!
\begin{lstlisting}[language=Python]
# Two Nested "For" Loops to illustrate the Cartesian Product
cartesian_product = []
A = [k,l,m]
B = [q,r]
for a in A:
    for b in B:
        ordered_pair = (a,b)
        cartesian_product.append(ordered_pair)

print(cartesian_product)  # [(k,q), (k,r), (l,q), (l,r), (m,q), (m,r)]
\end{lstlisting}
\par

It is possible for one factor of a Cartesian product to be a Cartesian product too.
For example, $\RR \times (\NN \times \ZZ) = \{(x, (y,z)) : x \in \RR, (y,z) \in (\NN \times \ZZ)\}$.\par

So far, we have been seeing Cartesian products between two sets, and we can also have Cartesian products with more than two sets.
Consider three sets, $\RR, \NN, \ZZ$. We will have a Cartesian product that has ordered triples.
$\RR \times \NN \times \ZZ = \{(x,y,z) : x \in \RR, y \in \NN, z \in \ZZ\}$.\par

In general, with $n$-sets:

\[A_1 \times A_2 \times \dots \times A_n = \{(x_1,x_2,\dots,x_n) : x_i \in A_i \text{ for each } i = 1,2,\dots,n\}\]
Do take note of the placement of the parentheses when reading Cartesian products.
While $\RR \times \NN \times \ZZ$ and $\RR \times (\NN \times \ZZ)$ look the same, they really aren't the same.
The first Cartesian product is between three sets while the second Cartesian product is between two sets.\par

For any set $A$ and positive integer $n$, the \textbf{Cartesian Power} $A^n$:

\[A^n = \underbrace{A \times A \times \dots \times A}_{n \text{ times}} = \{(x_1,x_2,\dots,x_n) : x_1,x_2,\dots,x_n \in A\}\]
the Cartesian power is more common than we think. The most common example so far is the case where $n = 1$ but we also see $n = 2$, and $n = 3$.
In particular, we can think of $\RR^2$ as the Cartesian plane that contains all the real numbers and $ZZ^2$ as a grid of points on the Cartesian plane.
Physically, we can treat $\RR^2$ as a piece of paper, and $ZZ^2$ are solid dots on the paper.
We can now draw links to coordinate geometry where we treat points in the forms of ordered pairs.
In this vein, we can say that $\RR^2$ is the combination of all possible coordinates on a real plane,
and $\ZZ^2$ is the combination of all coordinates whose elements are all integers.
For the case of $n = 3$, we can think of $\RR^3$ as the 3D space, and $\ZZ^3$ as a grid of points in the 3D space.\par

Try sketching a few of the Cartesian powers such as $\ZZ^2, \ZZ^3, \NN^2, \NN^3$ to see it for yourself!\par

\begin{figure}[H]
    \begin{minipage}{0.45\textwidth}
    \centering
    \includegraphics[scale=0.45]{./images/Z2-in-R2.png}
    \caption{$\ZZ^2$ in $\RR^2$ (Made in Desmos)}
    \end{minipage}\hfill
    \begin{minipage}{0.45\textwidth}
        \centering
        \includegraphics[scale=0.45]{./images/Z3-in-R3.png}
        \caption{$\ZZ^3$ in $\RR^3$ (Made in Desmos)}
    \end{minipage}
\end{figure}



\section{Subsets}
\section{Power Sets}
\section{Union, Intersection, Difference}
\section{Complement}
\section{Venn Diagrams}
\section{Indexed Sets}
\section{Sets That Are Number Systems}
\section{Russell's Paradox}
\end{document}